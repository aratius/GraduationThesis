\documentclass[a4paper]{article}
\setlength{\parindent}{1em}  % 一文字分インデント
\usepackage[utf8]{inputenc}
\usepackage{indentfirst}
\title{Imminence}
\author{Arata Matsumoto}
\date{April 2023}
\begin{document}
\maketitle
\section{Introduction}
  死の概念は、古来より芸術や哲学において重要なテーマとなっており、内省、自己発見、創造的表現の豊かな源泉を提供してきました。私たちは生きることによって死を受け入れる必要性を感じていますが、それは同時に恐怖や不安も引き起こします。当作品は、死と向き合う試みとして、現代美術の視点から新たな解釈を提案します。

  この文章は、死に対する日常的な遭遇と、それが個人の感情や思考に与える影響を描写することにより、死という普遍的なテーマを掘り下げています。作者は、展望台や海の堤防沿い、崖などの危険な場所にいることで、自身が簡単に死に至ることができるという不思議な感覚に取り憑かれることを語ります。この感覚は、「安全な臨死体験\footnote{著者による造語。死を簡単に選択できる状況を自認している際の心的状況を指す。}」として表現され、死が現実味を帯びて身近に感じられる瞬間を捉えています。

  また、作者は日常生活の中で死と向き合うさまざまな状況を紹介しています。通学時代の電車通勤やプログラマとしての仕事を通じて、死に対する恐怖や魅力を感じることが描かれています。特に、コンピュータを破壊する可能性のあるコマンドを入力することによって、再び死と向き合う機会を得たというエピソードは、現代のテクノロジーと死との関係を示唆しています。

  この作品は、死に対する認識や感覚を共有することで、読者に深い内省の機会を提供することを目指しています。読者は、作者が経験した感情や状況に共感し、自身の死に対する考え方や感じ方を再評価することが期待されます。さらに、作品を通じて得られる死と向き合う新たな視点は、読者が人生をより意識的に生きるきっかけとなるかもしれません。そして、人間の存在と死との関わりをより深く理解し、日常生活の中での死の意識を持続的に維持することにつながるでしょう。

  現代社会において、テクノロジーの進歩や情報の過剰な流れによって、私たちは死と向き合うことから遠ざかっているかもしれません。しかし、「安全な臨死体験」という作品は、現代美術の表現を通じて、死と向き合うことの重要性を再認識させる力があります。死を意識することは、人生の価値や意味を見つめ直す機会を与えてくれるだけでなく、人間の存在の根本的な問いについて考えるきっかけを提供します。

  この作品は、現代美術の枠組みの中で、死という普遍的なテーマを探求し、それを通じて人間の存在や意識についての理解を深めることを目指しています。また、作者の経験や感情を率直に綴ることで、読者が自らの死に対する感覚や意識を見つめ直す機会を与えています。

  さらに、作品は、死を恐れるばかりでなく、それを受け入れることで人生を豊かにする可能性を示唆しています。死を意識することで、自分自身や他者とのつながりを大切にし、人生の一瞬一瞬を大切に生きることができるというメッセージが込められていると言えるでしょう。

  最後に、この作品は、読者に死と向き合う勇気と機会を提供し、人生と死との調和を模索することを促します。現代美術における「安全な臨死体験」という作品は、死という普遍的なテーマを探求することで、人間の存在や意識についての新たな理解を生み出す可能性を秘めています。

\section{References}

\end{document}